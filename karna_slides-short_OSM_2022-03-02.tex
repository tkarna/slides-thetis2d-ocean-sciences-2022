%%%%%
%%%%% Template for beamer presentation using FMI 2019 style
%%%%%
\documentclass[
    presentation,aspectratio=1610,
    hyperref={colorlinks,citecolor=blue,linkcolor=black,urlcolor=blue},
]{beamer}
% \documentclass[presentation]{beamer}

%% Use beamer FMI theme
\useinnertheme{rounded}
\usetheme{FMI}

% \subtitle{my subtitle}  % add subtitle
%\institute[IL]{Ilmatieteen laitos} % change institute

%\setbeamertemplate{footline}[numberonly] % only page number in footer
%\setbeamertemplate{footline}[empty] % only page number in footer

%% Select language
%\RequirePackage[finnish]{babel}
\RequirePackage[english]{babel}

%% Input coding, Latin1:
%\RequirePackage[latin1]{inputenc}
%\RequirePackage[T1]{fontenc}
% or UTF-8:
\RequirePackage[utf8]{inputenc}
\RequirePackage{fontenc}

\RequirePackage{graphicx}
\graphicspath{{./fig/}} % graphics searched from fig directory
\RequirePackage{amsmath}
\RequirePackage{bm}

\RequirePackage{appendixnumberbeamer}

\usefonttheme{professionalfonts}

%% Fonts, Helvetica, with Euler math
\RequirePackage{eulervm}  % Euler math
\RequirePackage[scaled]{helvet}

%% Multimedia support
\RequirePackage{multimedia}
% \RequirePackage{media9}

\setbeamercolor{block title}{bg=fmiblue,fg=white}
\setbeamercolor{block body}{bg=fmiblue!10,fg=black}

%% Title and Author
\title[]{Adjoint-based data assimilation of sea surface height for the Baltic Sea}
\author[T. Kärnä]{Tuomas Kärnä\inst{1}, Joe Wallwork\inst{2}, Stephan Kramer\inst{2}}
\institute[]{\inst{1} Finnish Meteorological Institute \\ \inst{2} Imperial College London}
\date{\vspace*{6mm} Ocean Sciences Meeting, March 2, 2022}

\definecolor{crimson}{HTML}{DC143C}

\begin{document}

%% Main title page, automatically generated, use outside frame or make your own title page
\maketitle

\frame{
\frametitle{Adjoint-based inverse modeling}
\Large
\begin{enumerate}
 \item Automatic derivation of the adjoint model
 \item Application to parameter estimation:\\
 Optimizing bottom friction coefficient in a Baltic Sea simulation
\end{enumerate}
}

{\bgempty
\frame{
\frametitle{Automated generation of an adjoint model}
\begin{center}
 \includegraphics[width=0.7\textwidth]{figure_adjoint_model}
\end{center}
\vspace*{-3mm}
 \begin{itemize}
  \item Domain Specific Language modeling frameworks
  \item Model equations defined with a symbolic language (high level of abstraction)
  \item Automated code generator generates efficient C code at run time (low level)
  \item Adjoint model derived by differentiating the symbolic equations\\ and using the same code generator
  \item[+] Exact discrete adjoint model
  \item[+] Computationally efficient implementation
  \item Better than traditional \emph{automatic differentiation} that operates on source code level
 \end{itemize}
\vspace*{3mm}
{\scriptsize Farrell et. al (2013). Automated Derivation of the Adjoint of High-Level Transient Finite Element Programs.
SIAM J. Sci. Comput., 35(4), C369--C393. \url{https://doi.org/10.1137/120873558}}.
}
}

{\bgempty
\frame{
\frametitle{Firedrake framework/ Thetis model}
% \vspace*{-6mm}
\begin{columns}
 \column{0.50\textwidth}
 \includegraphics[width=\textwidth]{firedrake_logo}\\
 \url{firedrakeproject.org}
 \begin{itemize}
  \item Generic finite element modeling package
  \item Python embedded domain-specific language
  \item Automated adjoint capability
  \item \texttt{pyadjoint} package: tape etc.
  \item Efficient: Gradient evaluation cost\\ $\sim$4x forward model cost
 \end{itemize}
 \column{0.45\textwidth}
 \includegraphics[width=0.25\textwidth]{thetis_logo_white}
 Thetis ocean model\\
 \url{thetisproject.org}
 \begin{itemize}
  \item Unstructured mesh ocean model
  \item Implemented on Firedrake
  \item Discontinuous Galerkin FEM
  \item Both 2D and 3D versions
  % \item Adjoint capability
 \end{itemize}
\end{columns}
\vspace*{9mm}
{\scriptsize Kärnä et al. (2018). Thetis coastal ocean model: discontinuous Galerkin discretization for the three-dimensional hydrostatic equations. Geosci. Model Dev., 11, 4359--4382. \url{https://doi.org/10.5194/gmd-11-4359-2018}}.
}
}

\frame{
\frametitle{Application: Baltic Sea water elevation model}
% \vspace*{-6mm}
\begin{columns}
 \column{0.45\textwidth}
 \begin{itemize}
  \item Thetis 2D shallow water model
  \item Covers North Sea and Baltic Sea
  \item Fully implicit solver
  \item Efficiency: 1800x real time\\
        1 day in 48 s; 1 month in 24 mins
  \item North Sea: Tidally-dominated
  \item Baltic Sea: No tides, seiche oscillations, slow mean level variability
  \item SSH observations for 60+ tide gauges
 \end{itemize}
 \column{0.55\textwidth}
  \hspace*{-4mm}
  \includegraphics[width=1.1\textwidth]{domain_small}
\end{columns}
}

% \frame{
% \frametitle{Model configuration}
% % \vspace*{-6mm}
% \begin{columns}
%  \column{0.48\textwidth}
%  \begin{itemize}
%   \item Unstructured mesh\\ resolution: 500 m .. 13 km
%   \item Harmonie 2.5 km atmospheric forcing,\\ MSL pressure, 10 m winds
%   \item Tides from global tidal model\\ (TPXO 9)
%   \item River discharge (climatology),\\ 400+ rivers
%   \item SSH comparison at 63 tide gauges
%  \end{itemize}
%  \column{0.53\textwidth}
%   \hspace*{-5mm}
%   \includegraphics[width=1.1\textwidth]{mesh_014_stations_baltic_small}
% \end{columns}
% }
%
% \frame{
% \frametitle{SSH Animation}
%  \vspace*{-4mm}
%  \begin{center}
%  \includegraphics[width=0.95\textwidth]{animation_elev_m014_01_screenshot}
%  % \movie[autostart,loop,open]
%  %   {\includegraphics[width=0.95\textwidth]{animation_elev_m014_01_screenshot}}
%  %   {animation_elev_m014_01.mp4}
%   % \includemedia[
%   %       width=13cm,height=7.2cm, %
%   %       activate=pageopen,
%   %       keepaspectratio,          % optionally useful
%   %       playbutton=plain,
%   %       addresource=animation_elev_m014_01.mp4,
%   %       flashvars={
%   %           source=animation_elev_m014_01.mp4
%   %           &autoPlay=true
%   %           &loop=true}
%   %   ]{}{VPlayer.swf}
%  \end{center}
% }

\frame{
\frametitle{Optimizing bottom friction}
 \vspace*{-6mm}
 \begin{itemize}
  \item Cost function:
\begin{equation*}
 J(\mu) = \frac{1}{N_j N_i} \sum_{i,j} \frac{1}{Var(o_j)}\big((m_{j,i} - \overline{m_j}) - (o_{j,i} - \overline{o_j})\big)^2 + J_{reg}
\end{equation*}
  \item Mean-Square-Deviation, Bias removed, Scaled by observation variance
  \begin{itemize}
   \item $o_{j,i}$, $m_{j,i}$: observation/model time series, $j$ stations, $i$ time steps
   \item $\overline{m_j}$ time average
  \end{itemize}
  \item $J_{reg}$ additional regularization term
  \item Control variable: Manning friction coefficient $\mu$
  \item Adjoint model provides the gradient $d J /d \mu$\\
        $\Rightarrow$ can use any gradient-based optimization method
 \end{itemize}
}

% \frame{
% \frametitle{Regularization}
%  \vspace*{-6mm}
%  \begin{itemize}
%   \item Regularization term:
% \begin{equation*}
%  J_{reg}(\mu) = \alpha \|\mathbf{H}(\mu)\|_{2,1}^2 h^4 / A
% \end{equation*}
%   \item Squared norm of the Hessian matrix (2nd derivatives)
%   \item $h$ is the local mesh element size, $A$ is total the mesh area
%   \item Scaling $\alpha=400$ chosen to achieve a smoothly varying Manning coefficient ($\mu$) field
%  \end{itemize}
%  \vspace*{5mm}
% \emph{Best way to avoid over-fitting is to use a sufficiently long optimization period}
% }

\frame{
\frametitle{Optimization results}
\small
\begin{columns}
 \column{0.45\textwidth}
Optimized with a Quasi-Newton iteration (L-BGFS)
\begin{itemize}
 \item Period: June 1 -- June 17, 2019
 \item Initial Manning coeff. = 0.03
 \item 40 iterations
\end{itemize}
Optimized Manning field
\begin{itemize}
 \item Large variability in the North-Sea; \\ $\rightsquigarrow$ Propagation of tides
 \item Lower friction in the Danish Straits; \\ $\rightsquigarrow$ Volume flux to Baltic Sea
 \item Higher friction in the Archipelago Sea; \\ $\rightsquigarrow$ Unresolved archpelago
\end{itemize}
 \column{0.55\textwidth}
 \hspace*{-4mm}
 \begin{overlayarea}{\textwidth}{0.9\textheight}
 \includegraphics[width=0.65\textwidth]{progress_J_func} \\
 \hspace*{-8mm}
 \includegraphics[width=1.21\textwidth]{manning_field_i40}
 \end{overlayarea}
\end{columns}
}

\frame{
\frametitle{Validation}
\framesubtitle{Optimization improves skill significantly}
\vspace*{-10mm}
\begin{columns}
 \column{0.4\textwidth}
    3 month simulation: \\
    June – August, 2019.
    \begin{itemize}
     \item Centralized Root Mean Square Deviation (CRMSD)
     \item Baltic Sea
     \begin{itemize}
      \item $< 6\ \text{cm}$
      \item typ. $\sim3\ \text{cm}$
     \end{itemize}
     \item Danish Straits (DS + Kat, $\star$)
     \begin{itemize}
      \item $< 8\ \text{cm}$
     \end{itemize}
     \item Standard deviation (STDDEV) close to observation {\footnotesize (black markers)}
    \end{itemize}
    \vspace{2mm}
    \emph{Overall performance comparable to best forecast models}
 \column{0.6\textwidth}
%  \vspace*{9.7mm}
 \begin{overlayarea}{\textwidth}{1.0\textheight}
 % \includegraphics<1>[width=0.925\textwidth]{stats_slev_bswl-m014-01-bswl-m014-28-i40_2019-06-03_2019-08-31}
 \includegraphics<1>[width=\textwidth]{taylor_slev_bswl-m014-01-bswl-m014-28-i40_2019-06-03_2019-08-31}
 \end{overlayarea}
\end{columns}
}

\frame{
\frametitle{Validation: time series examples}
\small
\vspace*{-5mm}
\hspace*{-7mm}
 \alt<2>{
 \includegraphics[width=1.1\textwidth]{ts_Korsor_d0m_slev_2019-07-08_2019-07-19}\\
 Correct tidal amplitude in the Danish Straits.
 }{
 \includegraphics[width=1.1\textwidth]{ts_Degerby_d0m_slev_2019-06-01_2019-08-31}\\
 Correct wind-driven volume changes in the Baltic Sea.
 }
}

\frame{
\frametitle{Take-home messages}
\Large
\begin{enumerate}
 \item Automatic derivation of the adjoint model \\
 {\small
 \begin{itemize}
  \item Domain specific language modeling frameworks offer flexibility and computational efficiency
  \item The adjoint model can be derived automatically
 \end{itemize}

 }
 \item Application to parameter estimation:\\
 Optimizing bottom friction coefficient in a Baltic Sea simulation \\
 {\small
 \begin{itemize}
  \item Adjoint model is accurate and efficient
  \item Optimized model delivers excellent performance
 \end{itemize}
 }
\end{enumerate}
}

% \frame{
% \frametitle{Key take-home messages}
% \emph{Adjoint-based parameter optimization works}\\[3mm]
% Model configuration must be sufficiently good
%  \begin{itemize}
%   \item Mesh, bathymetry, forcings, initial condition\\
%         $\Rightarrow$ error is dominated by the control variable (Manning coeff.)
%  \end{itemize}
% \visible<2->{
% Cost function must be chosen carefully
%  \begin{itemize}
%   \item Measures the physical process of interest
%   \item All stations should have equal weight
%   \item Data preprocessing (whitening: bias removal and scaling)
%  \end{itemize}
% }
% \visible<3->{
% Regularization
%  \begin{itemize}
%   \item Sufficiently long optimization period (17 days)
%   \item Additional penalization of 2nd derivatives (Hessian)
%  \end{itemize}
% }
% }

% \frame{
% \frametitle{Conclusions}
% 2D water elevation model with an automatically-generated adjoint model
% \begin{itemize}
%  \item Results are encouraging
%  \item Adjoint model is accurate and efficient
%  \item Optimized bottom friction: excellent performance
% \end{itemize}
% Future work
% \begin{itemize}
%  \item Estimate initial condition (data assimilation)
%  \item Error quantification
%  \item Better optimization methods
%  \item Extension to Thetis 3D model
% \end{itemize}
% \vspace*{6mm}
% \emph{If you have any questions do not hesitate to contact me!\\
% {\footnotesize\href{mailto:tuomas.karna@fmi.fi}{tuomas.karna@fmi.fi}}}
% }

% \frame{
% \frametitle{Nemo-Nordic 2.0 operational marine model}
% % \vspace*{-6mm}
% \begin{columns}
%  \column{0.48\textwidth}
%  \begin{itemize}
%   \item Covers the North Sea and Baltic Sea
%   \item 1 nmi ($\sim$1.8 km) horizontal resolution
%   \item 56 vertical levels, $z^*$ partial cells
%   \item Surface resolution: 1 m
%   \item NEMO version 4.0
%  \end{itemize}
%  \column{0.53\textwidth}
%   \includegraphics[width=\textwidth]{domain_full}
% \end{columns}
% \vspace*{3mm}
% {\footnotesize
% Kärnä, T. et al., Nemo-Nordic 2.0: operational marine forecast model for the Baltic Sea, Geosci. Model Dev., 14, 5731–5749, \url{https://doi.org/10.5194/gmd-14-5731-2021}, 2021.
% }
% }

\end{document}
